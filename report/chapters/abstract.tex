\begin{abstract}

The aim of this project is to design and implement an \textit{Intelligent Assistant}, whose objective will be to recognise and identify the face of a student in order to provide personalised support. This support will consist on showing the timetable of the student and the route for their next class. The target of this project are students who have recently come to the University of Limerick, mainly Erasmus students. There are multiple approaches on how to develop face recognition, but in this case we have chosen Convolutional Neural Networks, as they are currently the state-of-art in face recognition, and in order to work with them, we are going to use the open source library \textit{TensorFlow}. \\
\\
However, the calculations done by this library need a computer with a certain power and therefore a considerable size. To provide mobility to the intelligent assistant and achieve a lightweight structure, the system is divided in two independent applications. First, a small, low-cost and basic computer named \textit{Raspberry Pi}, is going to be used to get the data from the camera and act as the user interface. This Raspberry will communicate as a client with the second part of the application: a Flask Server, allocated in another machine with higher specs. In this machine, we will use the data captured by the camera and the TensorFlow library to procced with the facial recognition and, in the final stage, to find the timetable of the student and the route to their next class. 

\end{abstract}











