\begin{abstract}

This Final Year Project describes in detail the development of an \textit{Intelligent Assistant}, whose objective will be to identify students of the University of Limerick using face recognition in order to provide them a personalised support. This support consists of showing what their next class in the day will be, hence the need to know the identity of the student, and then where will it be, showing a map with a path connecting the current position of the student to the location of the next class. The interaction with the \textit{Intelligent Assistant} will be through a graphical interface run by a small, low-cost and basic computer called Raspberry Pi. Due to the low computation power of this device, the application executed in the Raspberry Pi will communicate with a server located in another machine, which will actually perform the diverse tasks of the \textit{Intelligent Assistant}.\\
\\
The chosen approach to accomplish the face recognition are Convolutional Neural Networks, which have been widely used for image and video recognition in the recent years. Currently, one of the best face recognition implementations is FaceNet, with an accuracy close to 99.65\% in the lastest tests. In their Github website they have provided several pre-trained models since the presentation of the paper, among which a Inception Resnet v1 model was chosen to be used in this project. The Tensorflow library for Python 3 was resposible to load and train this model with a new dataset to carry out the task of face recognition.

%The aim of this project is to design and implement an \textit{Intelligent Assistant}, whose objective will be to recognise and identify the face of a student in order to provide personalised support. This support will consist on showing the timetable of the student and the route for their next class. The target of this project are students who have recently come to the University of Limerick, mainly Erasmus students. There are multiple approaches on how to develop face recognition, but in this case we have chosen Convolutional Neural Networks, as they are currently the state-of-art in face recognition, and in order to work with them, we are going to use the open source library \textit{TensorFlow}. \\
%\\
%However, the calculations done by this library need a computer with a certain power and therefore a considerable size. To provide mobility to the intelligent assistant and achieve a lightweight structure, the system is divided in two independent applications. First, a small, low-cost and basic computer named \textit{Raspberry Pi}, is going to be used to get the data from the camera and act as the user interface. This Raspberry will communicate as a client with the second part of the application: a Flask Server, allocated in another machine with higher specs. In this machine, we will use the data captured by the camera and the TensorFlow library to procced with the facial recognition and, in the final stage, to find the timetable of the student and the route to their next class. 

\end{abstract}
