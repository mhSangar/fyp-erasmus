\chapter{Description of Current Progress}
\label{current-progress}

%\section{Python Course}
%\section{Flask Server on Raspberry}
%\section{LaTeX Document}			% structure, packages, using Sublime Text 3...
%\section{Tensor Flow Tutorial}


%\section{Python Course}	% I don't think it will be useful
\section{Flask Server on Raspberry}
\section{LaTeX Document}			% structure, packages, using Sublime Text 3...
\section{Tensor Flow Tutorial}

