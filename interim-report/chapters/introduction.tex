\chapter{Introduction}
\label{introduction}

\section{General Introduction}
Humans have used body characteristics such as face, voice or even gait for thousands of years to recognize each other. Back in the 19th century Alphonse Bertillon (chief of the criminal identification division of the police department in Paris, France) laid the foundation of biometrics by his idea of using a number of body measurements to identify criminals. However, his idea was finally obscured by a far more practical discovery: the distinctiveness of the human fingerprints \cite{jain_biometrics}. 
time has passed, technology advancedd to identify by itself, no humans needed. in this project, face recognition. this 


Time has passed since that moment and technology has advanced enough to offer new ways to identify a person. 


face recognition was unfortunately limited severely by the technology of the era and computer processing power. However, it was an important first step in proving that face recognition was a viable biometric.

%\section{Objectives}
%\section{Scope}	% to explain what's in and out the project.
\section{Technologies Used}	% enumerate with the steps followed
	\subsection{Why Raspberry?}
	\subsection{Why Python?}
	%\subsection{Why Flask?}
	%\subsection{Why SQL-Alchemy?}
	\subsection{Why TensorFlow?}
\section{Motivation}
